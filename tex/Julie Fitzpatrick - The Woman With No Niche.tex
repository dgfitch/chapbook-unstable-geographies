{The Woman With No Niche}

by Julie Fitzpatrick 2020

Michelle was morose.

"What's wrong, my Shelly?" Her mother looked up from her design board,
glasses perched on her paint-spattered nose.

"You have your art and Daddy has his lab, but what do I have? I want to
make my mark, be successful, contribute. But where do I start? I love
people. I love lots of things. But no one thing stands out. I want to
learn everything, but the more I learn, the more directions I discover
waiting for me to explore them. How will I ever decide, Mom?"

Mom stopped painting, became thoughtful. "You realize you are very
talented in so many ways, don't you? I mean... you don't feel like you
are inadequate, do you?"

"No, Mom. No.~I just... I feel like... How did you know when you started
painting that that was what you wanted to do?"

"I didn't. I knew I liked to paint, but I never thought I would end up
doing it forever. It just happened. Sometimes things just happen. Your
Uncle Leonard, on the other hand, he knew when he was six that he would
study astronomy. Why don't you interview him and see what his take is on
the whole finding a direction thing?"

So she did. It went well. It went well enough that she decided to
interview as many people in professions that interested her as she
could. Surely one of these interviews would pique her interest in an
area of knowledge that she would want to pursue in depth.

"How are the interviews going, my Shelly?" Mom was tossing dye pods into
the raised vat where she created batik fabrics. She insisted that
fabrics with personal stories written on them with wax added power to
the creations that she made from them. Michelle couldn't argue, as her
mother's works were beyond beautiful. Michelle treasured the ones she
had been given.

"Everyone has been so kind. I've found that most people are eager to
share their passion for learning, and if I shape the questions right, I
get very thoughtful answers and honest self-analysis." She didn't bother
telling her mother that she had researched interview techniques to
become a better listener also.

She had found that if she left time for her subjects to think, they
would formulate new thoughts and feelings about the topic right there
during the interviews - not just react to old questions with old
answers. 'How did that make you feel? Why do you think you did that?
When did you first notice yourself thinking that? Where did you find the
motivation to try that?' She loved finding questions that nourished new
thoughts.

She found herself sharing personal insights with her subjects and being
gifted with knowledge in return. Each interview gave her new
appreciation for that person and for their area of expertise. But the
knowledge she gained was often personal, unrelated to that person's
given field. Michelle came to realize that the harder she listened, the
more the knowledge imparted seemed to pertain to deeper, intimate
knowings, things one dreams about and then forgets upon awakening. She
began interviewing anyone she could, regardless of whether they excelled
in a known area of interest to her. She found that everyone had
enlightening thoughts if she listened hard enough.

One day, her mother brought her Caring Tea and sat down with her on the
dais. "Shelly, I'm worried that you are spending all this time
researching the paths others have found to Fulfillment, but are coming
no closer to finding your own."

Michelle reached across the knee-hi serving block and placed her hand
over her mother's. "Why does this cause you such concern?" Her eyes
looked patiently into her mother's, searched her mother's face for
clues, waited for her mother to process the question. They sipped their
tea.

After a few moments tears formed in her mother's eyes. She blinked them
away and forced a wobbly smile. "When I was your age I wanted to become
an architect - to design buildings and communities that would honor the
earth and encourage the flourishing of our species." Her smile turning
self-deprecating. "But I met your father and contented myself with
lesser projects, so I could carry the weight of family and homemaking."
She took another sip of tea. So did Michelle. "I have no regrets. But
sometimes I wonder who I would have become, had I chosen my path just a
little sooner."

Michelle said nothing. She sat very quietly and waited, sipping her tea,
her attention on her mother, but not with a sense of expectancy; rather
in a relaxed, interested state.

"I realize now that my anxiety reflects that wondering." She poured more
tea into their cups. "I see that you have found your calling and I am
relieved that I can embrace it wholeheartedly."

Michelle frowned slightly, not following, but still said nothing.

"You are a Harkener - a Harkener of the human heart. You listen and you
hear; not just what is said, but what is waiting to be said - what is
felt, what is yearned for, what is in the heart."

Michelle felt a great weight lift from her shoulders. She and her mother
shared a hug, at peace with themselves and each other.

In The Garden, Michelle listened to the leaves as they fought to keep
their attachment to the great ash tree arching above. She realized she
had found her niche in the words of her mother. She had a gift, a skill,
a strength to build on now. Placing her ear against the rough bark of
the ash, she closed her eyes and listened.
