\emph{\ldots{} a man opens a wooden crate with the help of a bird
\ldots{}}

---J.M. Coetzee, ``Speaking in Tongues,'' \emph{The Australian}, 23
February 2006

He wanted to see what was inside his head, but his workshop had been
rifled by his disaffected spawn, and all that was left him in the way of
tools comprised a set of asymptotes, a spandrel, and something
resembling (but that he hoped was not; he had been rather relieved that
\emph{it} had not been taken) a choking-pineapple. Eventually, he
decided that the spandrel would suit his purposes best, and carefully
fitted it around the nape of his neck, just below his hairy ears, and
began to wind up his skull onto its verdigrised shaft. No matter how
meticulously he tried to keep the nutwings even, wrinkles formed, but he
hoped that a change in hairstyle might conceal them. The dura mater
parted like a veil of fears. Each time he inserted the spike of an
asymptote, spangles rose and fell, until the oil-stained workshop floor
was littered with their frozen corpses. He was getting somewhere at
last. He wistfully eyed empty chalk outlines where other tools had been
hung: scunner, martingale, castor, quetzal, bustier (\emph{that} one
would have come in handy for facilitating retention), figment, vortex.
He tried to look everywhere but at the pineapple. It felt not so much
that night was falling as that he was rising, slowly, through muffled
air, up the face of an infinite obsidian cliff. Somewhere on the moon, a
crater full of dust transformed itself in blue, reflected light.

first appeared in \emph{The Pedestal}
