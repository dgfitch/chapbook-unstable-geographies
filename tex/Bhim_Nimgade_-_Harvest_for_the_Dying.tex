\chapstory{Bhim Nimgade}{Harvest for the Dying}

\lettrine{O}{ur team has} streamed into the bleached zones, looking for the faded
blue of the dying eyes of the last creatures of a dying race. The
streaming, because nobody can go there in flesh bodies, is necessary,
but it is not perfect. Across the hot, hot sands, our human-type bodies
would require such expensive protection and infrastructure that the
funding groups would never acquiesce.

Still, those dying eyes must be found. Every one that sinks to the
heated ground and closes forever, to shrivel away, is a wasted treasure,
and opportunity. They have lived, and breathed, and thought, and
struggled, and then have come to the end. But yes, we can still glean
and gather what we can, and while we can.

The sky is pale and hot. Radiation pours down from the sky, visible
light of course, but also other ferocious frequencies. There is no sun,
no focal point, for these energies, but they come down on us from all
angles, from everywhere. There is no escape from their merciless glare.

Traversing a hillock, strewn with stones, it is hard to not to feel like
any step could be a wrong step, leading to a stumble, and then a fall to
perdition. But we are streaming. In a sense, we are not really there.

Our prey is not hiding. There is precious little cover, just endless
undulating lands with stones and sand and sandless domes of rock. But
those that we seek are small, in this vastness, and there are not many
of them.

Each life, I think, is a world; and a life lost, is a world lost. And so
I stride, virtually, across the land, searching, far from the others on
my team, but still connected, somehow. I don't see them, and they don't
see me, but we all trust that we are there, and connected, to each
other, and to our leader, and to Central..

There - a hint of something in the distance, something not a rock or
stone or sand. I pick up my pace. I turn down the contrast in my
visuals, turn down the brightness, and there it is. One of the dying.

It does not turn to look at me as I approach. I do not know, but I don't
think that there is anything for it to see, or to hear. It is dying -
they all are. And I am here, in time, to bear witness, and to do the
gleaning. It is still standing, supporting itself above the hot ground,
but it won't be for long, it seems. Its ungainly limbs seem to quiver,
like a distant mirage. I have to prepare for the capture.

Its head is dipping, and its eyes are closing. They open just a little
bit, sometimes, showing a touch of bleached and faded blue, and then
they start to close again. In the silence, so close to it, its breath
becomes audible, if it is indeed breathing. Something slow, and raggedly
rhythmical. There is so much light, and I inject it into my body, my
not-there body, and I let my arms rise up and embrace this creature. I
absorb into myself the images, the sense-impressions, the attachments,
the kaleidoscope of emotional states, the memories, the whirling
cacophony of wisdom and venality and kindness and brutality. It washes
over me and through me in a rush, too fast for me to experience any one
thing, but rather experiencing it all at once.

It was forever in a moment, a world in a grain of sand. I stepped back,
overcome.

``Go, then,'' I said. ``Your days have come to an end. You are walking
from the lands of the living, to another land. All those fetters which
have bound you to this existence are falling away.''

I had a vision, from another world far away, of bhikkhus with their
palms pressed together, chanting and bowing gently forward, welcoming
another creature to glide from one state of existence to the next. That
image, or memory maybe, has always helped to comfort and steady me, on
each mission. I shared it; though I do not know if it helped this lone
creature, in its passing, on this strange world far away.

I feel the call from Central. Time to depart. Even our streaming virtual
bodies have to flee this place. The world shimmers and dissolves, and
everything that I am attenuates into a stream of dancing particles, and
my final memory is the acrid smell of stardust.
