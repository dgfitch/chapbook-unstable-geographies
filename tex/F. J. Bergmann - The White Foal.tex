At that time, I lived with my grandmother in the Camargue, the marshy
delta where the Rhône river widens and opens into the Mediterranean sea.
My parents, whom I could almost no longer remember, were dead. They had
left no inheritance, but in truth, Grand-mère, who was skilled with
remedies, was not in any need. It was whispered in the marketplace that
she was a sorceress---but well-liked by the local folk who owed their
health or lives to her.

Grand-mère did not make me work very much. With the wisdom of great age,
she let me roam, carefree, on the marsh and its waters, along the
shores. She said, ``There will be time, when you are older, to learn
your duties. But now, now is the golden summer of your childhood.''

The water was my life. Grand-mère had taught me to swim right away.

``A good sorceress is always prepared for any eventuality,'' she assured
me.

She gave me a little boat, just for me. In my coracle, I navigated the
marshes, ponds and lagoons, to the deepest currents of the river and
along the seacoast.

But what I wanted more than anything was one of the wild horses of the
Camargue. Storm-gray or white as clouds without rain, I watched them
from far off, silhouetted against the horizon, passing like pale shadows
through the reeds, or swimming across the deep waters. They were also in
the marketplace: mastered, mounted, harnessed, tamed, broken. What I
desired was not that, but that one of the wild foals would befriend me,
of its own free will.

I had tried to pursue them on foot or in my coracle, but galloping or
swimming, the horses were too fast---I was never able to so much as
touch the end of a tail. They seemed made of magic, atmospheric, in the
same way as the rose-colored flamingoes and the herons that flew off in
a great rustle of wings no matter how silently I approached them.

One afternoon, I had wandered a great distance along the beach, without
a care---but Grand-mère had shown me a species of little spotted shell,
of great value for her medicines, and I wanted to please her. I had
found two of the shells; sunset was not far away. Returning to where I
had drawn up my boat, I saw a small white shape, almost silvery,
stretched out on the sand. \emph{A drowned foal---oh, no!} I thought,
and I began running toward it.

Weeping, I knelt at its side. But it still breathed. Its eyes opened.
Blue eyes, eyes of azure, like the sea, like the sky.

All the other horses I had ever seen had brown eyes.

I had not recovered from my surprise when the little colt spoke to me in
a soft, childlike voice.

``I am dying from the heat of the sun. Please, I beg you, pull me into
the cool sea.''

This was the companion of my dreams, but for an instant or two I
remained immoblie.

``Have pity on me; I am burning!''

I began tugging him toward the water. Gods, he was heavy\ldots{}. But I
persevered; the tide was rising and finally a wave caught and lifted
him, and he curled his long legs under him. He nosed me gently with his
soft muzzle, catching my eyes in his blue gaze.

``Mount on my back; we will swim away and be friends forever.''

He seemed to swell, as if he were growing larger moment by moment,
floating at his ease. Under the waves, I could see the glint of scales
beginning along his haunches: no longer legs, but a tail like a fish.

``I will take you to your parents.''

I knew his last promise was no lie. I also recognized his attempt to
bewitch me. I looked away and took care not to meet those blue eyes with
my own again. I dragged my boat further up into the dunes, where the
tide could not reach it, and went the long way home, on foot through the
marshes---I would have to be very cautious at the edge of the sea from
now on, but Grandmère would use her arts see to my safety.

Someday, I hope a very long time from now, when I finally want to die, I
will find the white colt once more.

first appeared in \emph{Abyss \& Apex}
