It had been more than eighteen months since Joseph had tasted beer. His
tongue reached up to lick the foam from his upper lip. He closed his
eyes and smiled gently as his hands cradled the plastic cup, taking care
not to spill a drop. He listened to the foreign-sounding music, noting
the festive air it evoked. He watched the shuffling clusters of people
gathered around each beer-dispensing window, their faces tired,
disbelieving, but hopeful as they waited patiently for their plastic cup
of golden goodness.

Joseph wondered about this strange departure from the usual order of
things, if there can be a usual order here. Since he had arrived at the
camp, beer had never been offered. As refugees, they were lucky to get
water and rice regularly. Anything else was rare. Beer? Unheard of. If
everyone weren't so tired, there would be a riot, with the more
ambitious trying to get more than their share.

The riot-geared guards atop the hatches of the tankers that were serving
as beer dispensaries may also have had something to do with the overall
good behavior - the guards, and the appearance of plenty. There looked
to be enough beer in the huge tankers to last for days. Giant painted
letters on their sides read FREE BEER, strangely grafittied in different
languages.

They stood out like giant buoys in the sea of hovels that stretched to
the horizon of this war-torn corner of the world, or strangely bloated
bugs, with hoses instead of legs, like spiders.

The lines grew longer as word of their presence spread. A makeshift
orange fence surrounded the tankers and their attendants and hoses, with
holes well placed, just large enough to accommodate the serving of the
beer. The music coming from the speakers atop the tankers was
happy-sounding, and as the day wore on, there was much dancing. But
eventually, all had their fill. Available containers had been filled and
tucked away for tomorrow. Lines had dwindled to the occasional wobbly
repeater. Finally, the orange fence came down and the tankers maneuvered
themselves into a convoy and left. Joseph returned to his home spot,
spread his blanket, and slept without dreaming.

This scene replayed in refugee camps worldwide, then disappeared. The
'free beer days' were not reported to anyone in an official capacity.
Had they been, it would have been discovered that they were not part of
any sanctioned relief agency. No country sent them - nor did the U.N. or
the Red Cross or any other aid organization.

Within a month of their appearance - and disappearance - a new brand of
beer began an aggressive marketing campaign worldwide. FREE BEER hit the
liquor stores and grocery aisles in a carefully orchestrated
first-offering. Priced reasonably, with a wonderfully palatable flavor,
it appealed to most. It was well-received and enjoyed a large share of
the beer market in every country where it was sold, which was every
country, had anyone bothered to check. No one did.

Just about the time FREE BEER saturated the market - a household name in
record time - it disappeared from shelves everywhere without a trace.

A few beer drinkers grumbled, but most just went back to their old
favorites and forgot about FREE BEER.

Within a year, a strange phenomenon was noted. The number of new
pregnancies had dropped significantly. Almost overnight, female and
especially male infertility rates had skyrocketed. Scientists were
mystified. Humans had inadvertently caused the extinction of thousands
of species of both plants and animals, so experts had plenty of
experience tracing how these tragedies came about, but the onset of this
pandemic had been surreptitious and sudden. War loomed, as countries
blamed each other.

Eventually, the infertility was traced back to the beer. There's always
a stale bottle of beer tucked away in an old refrigerator in the
basement, waiting for a least favorite in-law or uninvited guest to
visit. Sure enough, the beer contained a genetic modifier that rendered
its drinker infertile - a birth control vaccine.

Very few children were born for nearly a generation - not until the
children who were too young to have tried FREE BEER were old enough to
procreate. With the precipitous drop in population, as natural attrition
was not balanced by new births, there was a shift in how people saw the
future. The growing children received more care and attention, better
education.

The culprits who spread the FREE BEER vaccine were never caught. It was
predicted they would strike again once the population rebounded to
catastrophic proportions, again requiring extreme intervention.

Some referred to the mysterious neuterers as terrorists, others as
saviors. Communities that had not partaken of the tainted beer kept a
low profile, lest they be blamed for the chicanery. The followers of
Islam and others who avoided alcohol were not spared, as the vaccine had
found its way into bottled water sources as well.

Leaders worldwide vowed not to waste the population reprieve, promising
better laws to limit climate change, improve air and water quality,
protect species diversity, pursue cleaner energy sources, keep food
sources genetically safe. They were the same old promises, surely, but
with a ring of sincerity. Maybe it was because now there was a chance
those goals could be achieved before it was too late. Maybe the next
generation, having the benefit of hindsight and hope, would make choices
that could be lived with. Or maybe next time, no one would be spared.
