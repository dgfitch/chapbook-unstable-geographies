My neighbor said he thought he'd build a wall; wanted to know if I'd go
halves on it. I asked him what he was going to make it out of and he
said ``Words,'' and I said I'd help him out as much as I could. I asked
him how high he was going to make it and he said ``High.''

He started out with long, Latinate words, at least five syllables,
carefully staggering the joints, but he ran out of his own almost right
away, so I had to give up a lot of mine. He tried to maintain a
structured form, but soon it degenerated into a random jumble, mostly
nouns and verbs---he was saving the adjectives to decorate it when it
was finished, he said, stacking them neatly against the porch. The
articles and conjunctions kept falling out and accumulated in forlorn
drifts at its base.

He worked on it every evening, after coming home from his regular job,
until night fell, late into the autumn. Joggers would occasionally stop
to offer advice and put in a word or two. It spread like a blackthorn
hedge above its massive foundation, tangling tightly as the barbed
serifs hooked together. The wind whistled through the small openings of
the a's and e's as the larger counters of the o's, b's, d's, p's, and
q's resonated at a lower pitch. He placed the sharpest words along the
top of the wall. ``Expect trouble,'' he said.

During the winter, the ascenders and descenders began to distort and
twine around letters in adjoining words. Just before the solstice, I
hung the most ornate plural nouns and third-person-singular verbs I
could find on the north side of the wall. Dangling from each terminal s,
they swung like bells, chiming as the snows fell. That spring, suffixes
sprouted from the side that faced the sun.

first appeared in \emph{Wisconsin Academy Review}
