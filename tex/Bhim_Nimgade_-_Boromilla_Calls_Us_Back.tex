\chapstory{Bhim Nimgade}{Boromilla Calls Us Back}

\lettrine{B}{oromilla sinks} deeper and deeper into the plush pillows and bolsters of
the ancestral bed, the luxurious ornate headboards and carved statuary
siderails in mute witness of her languor. The world itself has sunk
lower and lower into the twilight of times, a baleful glowing globe
pulsating fitfully and ebbing in its strength. There is no use in
strife, in anguish, in achievement anymore. The tiny glints of colony
ships are all winding homeward in the vastness of space.
It will take---or it has taken---centuries for all of them to return, but they are on
their way, to seek the solace of the mother planet for the last time.

There have been other worlds, as tiny microcosms of earth tore
themselves away in valiant colony ships to climb and explore and
colonize, fruitfully, through nearspace. Dwarf stars with huge ice
planets, and colonies burrowing under the frozen atmospheres to shelter
in the strange warmth oozing up from below. Colonies digging into the
crumbling icy dust of captured comets in stable orbits, just far enough
away from the central star to survive, living on the liquid water and
organic compounds in them. Not the wild erratic orbits of the
sun-grazing comets, spending eons in the outer reaches, cold and stable,
and then an unexpected perturbation nudging them, to plunging faster and
faster toward the central star, as its hot glory boils off the volatiles
into a dusty sweep of millions of kilometers of cometary tail, a tenuous
veil that wastes its substance and wafts it away into nothingness.
Colonies drifting in the interstellar vacuum, rudderless, knowing that
ahead of them lies only more emptiness, taking measures to prolong their
survival but knowing that it means nothing, there would be no rescue,
that their children's children would wake up someday to the certainty of
death. Ships that once rose, mightily, into the universe, and now have
gathered what resources they can to patch themselves together to limp
back home.

And Boromilla sinks back into her bed, wondering which of her children
she will see again, and will they come back to find her. Will they know
her when they reach her? Will they burn through the leagues of strange
and twisted trees, the endless rubble of faded plastics bobbing forever
on shallow seas, the bubbling mats of deadly algae, the fetor of dead
silvery bodies piled up on shores with their dulling sightless eyes
unable to close, feeble trickles of water streaked with unnatural colors
whispering their poisons into the land, citadels ancient and crumbling
or recently and proudly built but now crumbling into slopes of talus
with the marks of plasma cannons and bombs and eldritch fires.

Boromilla summons them to come home. And across hundreds of parsecs,
they obey, maybe not knowing what drives them to uproot themselves from
whatever soil, fertile or cruelly barren, they have landed upon. She
calls them, the great mother calls them, and obey they must.

The few who cannot heed her are the ones who have lost power, and go
blindly on, feeling a sick yearning to go back but they have no power to
do so; and those on gossamer light sails, slowly gathering speed to
spread outward, with no hope of changing their trajectories until they
reach another star system to capture them and draw them in.

Ah, her children had dreamed of a vast universe with verdant playing
fields and planetary ring worlds and undying fountains of energy, life
unleashed to spread and flower across the stars. Some colonies would
survive but then gutter out, menaced by some cold unfeeling slow
catastrophe; or they would blaze fiercely and die in some hot wild
cataclysm flung at them by the universe, or brought on by their own
unrelenting hatreds. Some few thrive and flourish and build wondrous
civilizations that would echo with song and give life to dreams.\ldots

Boromilla weaves the threads of all the worlds, and draws them to
herself. This is home. Home is in eclipse, darkening, dying. All must
come and pay tribute in these last days. Her thick blood glows inside
her body, moving more and more slowly. Colony ships arrive and are moved
into earth orbits, and abandoned there, as her children climb into
lifeboats to glide down to earth. Some, too far gone, just aim for the
surface, and they light up the night sky with blazing trails in their
fiery re-entry. Thunder cleaves the sky and oceans boil up in miles-high
fountains when they hit. Boromilla can feel their arrivals, sensing in
her body the arrivals of her children, however distant, and slow smiles
move over her sleeping face.
