The condition proved to be untreatable. Following the first austere
reports in the \emph{New England Journal of Medicine}, it spread slowly
westward, crossing the Mississippi seventeen months after the first
confirmed diagnosis. Initially, doctors assumed their patients were
concealing objects in their mouths, or swallowing and vomiting them up
at will, as a manifestation of adolescent hysteria (the first to be
afflicted were always teenage girls). When the volume exceeded any
imaginable human capacity, paranormal researchers began to take an
interest.

One enterprising pediatrician deliberately elicited from a young
patient, by means of lengthy oral interviews, several thousand faceted
gemstones, claiming that they were necessary for lab samples. All were
first-quality rubies and sapphires, which he sold on eBay. He was much
less attentive to her older siblings, who generally produced only
worthless pebbles, or, if greatly annoyed, toads and tadpoles. By the
time their parents realized his deception and lack of progress with the
condition, they too had become infected. The parents' case was
frequently cited by specialists to exemplify how emotions colored
formation of the objects in question; how their daughters' physician had
perished beneath a barrage of invective and small---but
poisonous---snakes.

Politicians were especially inconvenienced as their campaign promises
and rhetoric took tangible form, to the dismay or delight of supporters
and detractors alike. Particularly noisome ideologies became the
projectiles that drove their proponents from the platform. Comedians and
poets performed with mikes turned off, and were judged by the quality of
the material goods they produced for audiences to enjoy.

Coughing and sneezing from sufferers generated, respectively, moths and
butterflies. Some became voluntarily mute before any symptoms had the
opportunity to manifest. The form of whatever onerous ghosts or ominous
histories would have been expressed from the torque of their earlier
torments could not be inferred, their quiescent tongues burdened only
with silence.

But singers---and their enchanted listeners---gloried in the birds that
flew from their tongues. The fascinated fans of a great opera diva never
tired of the flights of swans that soared out from each of her sublime
arias.

first appeared in \emph{Lakeside Circus}
