\chapstory{Bhim Nimgade}{Faces Floating Up}

Imagine the whining of the grand engine and the winding down of
civilization. The mighty works of humanity, finding themselves without
foundation and substance, unraveling themselves.

I had a vision, of the last forlorn remnants of humanity, in tattered
scraps of clothing, gathered on a hill, looking up into the riven skies,
waiting for signs of divine or other grace.

And every face rises up from its place on an upturned skull, tearing
itself off of the bone, peeling itself free, to float in the air and
murmur and cry out for something. Justice? Recompense?

Startled, the bodies left behind jostle each other blindly, and strut in
ungainly array.

Pestilence strides over the dark land. Chaos crawls and writhes
underneath the boot.

The grand squares, the piazzas, the sun-dappled courtyards, are quiet
now. Civilization and progress are now memories. I can feel a sting of
tears in waiting, behind my dry and reddened eyes. Why wander these
desolate streets? You cannot bring them back. A furtive movement around
a corner - it's another wanderer, drifting, through these empty streets,
thinking of what once used to be. You can't go home again. Home is not
there anymore. Go to the place where you bed down, and you sit and gaze
gloomily at a wall as darkness comes. Too sullen rise up to go to bed,
too tired to stumble to the spot where civilization insists that you
must sleep when it is time. No, you sit there, quietly, as your limbs
grow heavier and you cannot rise. Some time in the dark, your heavy eyes
close, and when they open, it is morning, stiff and weary and unrested.

You remember her. How you want to call her, and tell her of what is
happening. It's the end of the world, but you don't want to tell her
that. You want her to ask you about what is happening, so you can soothe
her, you can tell her not to fear, tell her that we can get through
this. (Through the end of the world).

But you can't call her. There is no one to call.

After all, she left a long time ago. Long before all this. She would
have been terrified, when all the world fell silent. When people faded,
faded away. Those who had been present and real, became dimmer and
quieter. And further away. Much further away.

Could you have comforted her? What would you tell her? That this is the
way the world ends, this is just what happens, as time's arrow points
immutably to weariness and decay?

She left this world, back then. Was it kinder for her, that way of
passing? To see the world going on, as it should, but leaving her
somehow behind; and just her, herself falling, falling out of step with
this busy world, this increasingly incomprehensible world. Did she know,
as she felt herself slipping away, that this world would go on?

But for us, we have no such assurance. We see this world stumble, and
lose its surety and solidity. We had woven this world together, but the
weaving is dissolving. We are all slowing, and we are all sixpence
short.
