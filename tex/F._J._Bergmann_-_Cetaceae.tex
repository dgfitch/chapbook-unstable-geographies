\chapstory{F. J. Bergmann}{Cetaceae}

If I were to write concerning whales, I would not write about their
lessening numbers or their sadness, but their invisible harmonics and
how they disguise themselves as floating islands, pod archipelagoes.
Islands that sing; Kalypso had one of those.

They gather the secrets that flow away from continents on their drifting
shores, water them with spray from the reliable geysers of their
blowholes. Each island garden greens in the long salt years, tangling
threads of root and rhizome prickling through cetacean skin into
nourishing blubber, dangling into the sea. In the mat of trailing
branches, multitudes gather, creatures that look like the writhing
contents of an Italian restaurant menu.

The mass insanity of flowers casts away a wake of maddened petals that
wash up on the beaches of crabs, gulls, and plutocrats. Palm trees shoot
skyward from stranded coconuts; rising cypresses grasp the shoreline; a
baobab spreads its knotted arches; lianas thread toward sun. Thickening
jungle quickens with a pair of colobus, a flickering spectrum of lost
birds, an errant peccary, and one leaf-green snake.

Sinbad and his exhausted sailors beach their boat, fall to the warm
sand, find a fountain of clear water. Later, an aroma of roast pork and
monkey-meat lingers above the empty ocean.

But whales are warm like us, and if I wrote about whales I would mention
their vast journeys, the taste of plankton and giant cephalopods. I
would have to say something about the currents, the dark nights, the
harpoons, the cold.

first appeared in \emph{Weatherbeaten}
